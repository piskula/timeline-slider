%%%%%%%%%%%%%%%%%%%%%%%%%%%%%%%%%%%%%%%%%%%%%%%%%%%%%%%%%%%%%%%%%%%%
%% I, the copyright holder of this work, release this work into the
%% public domain. This applies worldwide. In some countries this may
%% not be legally possible; if so: I grant anyone the right to use
%% this work for any purpose, without any conditions, unless such
%% conditions are required by law.
%%%%%%%%%%%%%%%%%%%%%%%%%%%%%%%%%%%%%%%%%%%%%%%%%%%%%%%%%%%%%%%%%%%%

\documentclass[
  digital, %% This option enables the default options for the
           %% digital version of a document. Replace with `printed`
           %% to enable the default options for the printed version
           %% of a document.
  twoside, %% This option enables double-sided typesetting. Use at
           %% least 120 g/m² paper to prevent show-through. Replace
           %% with `oneside` to use one-sided typesetting; use only
           %% if you don’t have access to a double-sided printer,
           %% or if one-sided typesetting is a formal requirement
           %% at your faculty.
  notable,   %% This option causes the coloring of tables. Replace
           %% with `notable` to restore plain LaTeX tables.
  nolof,   %% This option prints the List of Figures. Replace with
           %% `nolof` to hide the List of Figures.
  nolot,   %% This option prints the List of Tables. Replace with
           %% `nolot` to hide the List of Tables.
  %% More options are listed in the user guide at
  %% <http://mirrors.ctan.org/macros/latex/contrib/fithesis/guide/mu/fi.pdf>.
]{fithesis3}
%% The following section sets up the locales used in the thesis.
\usepackage[resetfonts]{cmap} %% We need to load the T2A font encoding
\usepackage[T1,T2A]{fontenc}  %% to use the Cyrillic fonts with Russian texts.
\usepackage[
  main=slovak,  %% By using `czech` or `slovak` as the main locale
                %% instead of `english`, you can typeset the thesis
                %% in either Czech or Slovak, respectively.
  %english, german, russian, czech, slovak %% The additional keys allow
]{babel}        %% foreign texts to be typeset as follows:
%%
%%   \begin{otherlanguage}{german}  ... \end{otherlanguage}
%%   \begin{otherlanguage}{russian} ... \end{otherlanguage}
%%   \begin{otherlanguage}{czech}   ... \end{otherlanguage}
%%   \begin{otherlanguage}{slovak}  ... \end{otherlanguage}
%%
%% For non-Latin scripts, it may be necessary to load additional
%% fonts:
\usepackage{paratype}
\def\textrussian#1{{\usefont{T2A}{PTSerif-TLF}{m}{rm}#1}}
%%
%% The following section sets up the metadata of the thesis.
\thesissetup{
    date          = \the\year/\the\month/\the\day,
    university    = mu,
    faculty       = fi,
    type          = mgr,
    author        = Bc. Ondrej Oravčok,
    gender        = m,
    advisor       = {RNDr. Radek Ošlejšek, Ph.D.},
    title         = {Interaktívna časová os ako modul v Angulare},
    %TeXtitle      = {The Proof of $\mathsf{P}=\mathsf{NP}$},
    TeXtitle      = {Interaktívna časová os ako modul v Angulare},
    keywords      = {keyword1, keyword2, ...},
    TeXkeywords   = {keyword1, keyword2, \ldots},
    abstract      = {This is the abstract of my thesis, which can

                     span multiple paragraphs.},
    thanks        = {These are the acknowledgements for my thesis, which can

                     span multiple paragraphs.},
    %bib           = prace.bib,
}
\usepackage{makeidx}      %% The `makeidx` package contains
\makeindex                %% helper commands for index typesetting.
%% These additional packages are used within the document:
\usepackage{paralist} %% Compact list environments
\usepackage{amsmath}  %% Mathematics
\usepackage{amsthm}
\usepackage{amsfonts}
\usepackage{url}      %% Hyperlinks
\usepackage{markdown} %% Lightweight markup
\usepackage{listings} %% Source code highlighting
\lstdefinelanguage{TypeScript}{
    morekeywords={number, string, object,
        typeof, let, public, private, void
        },
    sensitive=false, % keywords are not case-sensitive
    morecomment=[l]{//}, % l is for line comment
    morecomment=[s]{/*}{*/}, % s is for start and end delimiter
    morestring=[b]' % defines that strings are enclosed in double quotes
}
\lstset{
%  basicstyle      = \ttfamily,%
%  identifierstyle = \color{black},%
%  keywordstyle    = \color{blue},%
%  keywordstyle    = {[2]\color{cyan}},%
%  keywordstyle    = {[3]\color{olive}},%
%  stringstyle     = \color{teal},%
%  commentstyle    = \itshape\color{magenta}}
  language=TypeScript,
  aboveskip=3mm,
  belowskip=3mm,
  showstringspaces=false,
  columns=flexible,
  basicstyle={\small\ttfamily},
  numbers=left,
  xleftmargin=1.25em,
  numberstyle=\tiny\color{gray},
  keywordstyle=\color{blue},
  commentstyle=\color{darkgray},
  stringstyle=\color{olive},
  breaklines=true,
%  breakatwhitespace=true,
%  tabsize=3
}
\usepackage{floatrow} %% Putting captions above tables
%\floatsetup[table]{capposition=top}
\usepackage[backend=biber,sorting=none]{biblatex}
\addbibresource{prace_mgr.bib}
\usepackage{graphicx}
\graphicspath{ {pics/} }
\usepackage{tabu}
\usepackage{multirow}

\begin{document}
\chapter*{Úvod}
\addcontentsline{toc}{chapter}{Introduction}
Toto bude úvod.

\chapter{Kybernetický polygón}
Podľa Národného bezbečnostného úradu Českej republiky je termínom \textit{kritická informačná infraštruktúra} označený \textit{"systém prvkov tejto infraštruktúry, ktorý pri narušení svojej funkcie môže spôsobiť poškodenie alebo ohrozenie záujmov Českej republiky"}\cite{nbu2012}.

Kybernetický polygón (KYPO) bol zriadený Ministerstvom vnútra Českej republiky za účelom zaistenia kybernetickej bezpečnosti v Českej republike ochranou kritickej informačnej infraštruktúry. Projekt KYPO poskytuje unikátne prostredie pre výskum nových metód, ktoré by v blízkej budúcnosti mohli pomôcť chrániť kritickú informačnú infraštruktúru\cite{dankovvcikova2015konfigurace}. Objekt polygónu bol vybudovaný v roku 2015 vrámci rekonštrukcie Fakulty Informatiky.
%V rýchlo sa rozvíjajúcom svete informačných technológií je informačná bezpečnosť najdôležitejším odvetvím informatiky. V dnešnej dobe už nie je veľmi pravdepodobné, aby nás výdobytky techniky ako telefóny a počítače ohrozovali na živote. Hrozba informačnej bezpečnosti sa týka hlavne ochrany informácií. Masarykova Univerzita si uvedomuje vážnosť tejto hrozby a preto sa rozhodla zriadiť výzkumné stredisko v tejto oblasti.

\section{Technológie}
Toto prostredie funguje na princípe cloudového riešenia v ktorom je možné modelovať virtuálnu sieť\cite{eichler2014analytical}. Toto riešenie poskytuje všetky nástroje potrebné k simulácii rôznych sieťových topológií. Prostredie je určené pre výzkum a testovanie rôznych scenárov kybernetických útokov v izolovanom a plne kontrolovateľnom prostredí\cite{vceleda2015kypo}.

Spektrum kybernetických hrozieb vo svete informačných systémov je veľmi široké a každá hrozba má iný spôsob, ktorým narúša kontinuitu iného systému. Medzi časté experimenty vykonávané vrámci KYPO stoja za zmienku DDoS útoky (distributed denial of service), útoky botnet alebo phishing\cite{vcegan2014navrh, celeda2013projekt}.

\section{Architektúra}
\begin{figure}
	\center
	\includegraphics[width=1.0\linewidth]{kypo_structure}
	\caption{Architektúra cloudového riešenia KYPO\cite{vceleda2015kypo}}
	\label{kypo_structure}
\end{figure}

Prostredie kybernetického polygónu je implementované ako cloudové riešenie modelu Platform as a Service. Obrázok \ref{kypo_structure} znázorňuje jednotlivé vrstvy architektúry:

\begin{description}
\item[Users] - užívatelia s rôznymi skúsenosťami interagujú so systémom buď cez Portál, alebo priamo s niektorou nižšou vrstvou
\item[Portal] - Portál je grafické používateľské rozhranie (GUI), ktoré v prípade KYPO využíva portálový server Liferay opísaný v kapitole~\ref{liferay}. Praktická časť tejto práce sa zaoberá tvorbou portletu práve do tejto vrstvy architektúry.
\item[Scenario and sandbox management API] slúži na konfiguráciu, vytváranie, editovanie a likvidáciu sandboxov. Sandbox je izolovaný súbor virtuálnych strojov a sieťovej konfigurácie, ktorý užívateľovi poskytuje kľúčovú funkcionalitu ako napríklad možnosť opakovať a modifikovať experiment v prostredí KYPO.
\item[Monitoring API] umožňuje monitorovanie sieťových prepojení a konfigurácie uzlov. Komunikuje buď s OpenNebula alebo priamo s existujúcimi virtuálnymi inštanciami.
\item[Cloud API] slúži ako univerzálne rozhranie pre vyššie vrstvy ktoré abstrahuje od konkrétnej implementácie IaaS\footnote{Infrastructure as a Service} nižších vrstiev, aby bola zabezpečená flexibilita cloudového riešenia.
\item[OpenNebula platform] implementuje IaaS\cite{sempolinski2010comparison}. Umožňuje manažment rôznorodých výpočtových kapacít (najčastejšie virtualizovaných) a spadá pod správu CERIT Scientific Cloud.
\item[Computing infrastructure] zahrňuje fyzické objekty, sieťové prvky a všetok potrebný hardvér, ktorý poskytuje pamäť a výpočtovú silu pre.
\end{description}

\section{Potreba analýzy dát}
Nakoľko simulácie útokov a cvičenia prebiehajú v reálnom čase, vznikla potreba sledovať zmeny v týchto dynamických prostrediach. Priebehy jednotlivých scenárov sa zaznamenávajú aj pre potreby neskoršej analýzy. Portlet časovej osy slúži ako kľúčový element pre filtrovanie týchto dát, či už sa jedná o analýzu starších dát, alebo sledovanie aktuálne pribúdajúcich dát počas práve prebiehajúceho experimentu.

\chapter{Liferay Portal}
\label{liferay}
V kontexte webových aplikácií môžeme portál definovať\cite{sezov2011liferay} ako \textit{"Samostatne oddelené webové prostredie, z ktorého sú spustiteľné všetky aplikácie užívateľa. Tieto aplikácie sú systematicky integrované na jednom mieste."}

Liferay je portálový server, v ktorom je zmysluplne umožnená spolupráca a zdieľanie informácií rôznym používateľom. Umožňuje vykonávať správu obsahu pomocou WYSIWYG\footnote{WYSIWYG - \textit{what you see is what you get} teda \textit{"to, čo vidíš je to, čo aj dostaneš"} je spôsob práce s grafickým rozhraním, pri ktorom sa kód (najčastejšie HTML) generuje na základe našich predstáv, pričom do neho vôbec priamo nezasahujeme} editora. Používateľ tak nepotrebuje žiadnu hlbokú znalosť HTML ani iného programovacieho jazyka. Jednotlivé komponenty stránky si vie jednoducho prispôsobiť a výsledok vidí vždy okamžite\cite{sezov2011liferay}.

Portálový server Liferay je naprogramovaný v jazyku Java a distribuovaný pod licenciou LGPL. Na jeho beh potrebujeme aplikačný server, alebo aspoň servlet kontajner. Liferay natívne podporuje napríklad Apache Tomcat, GlassFish, JBoss, WebLogic, WebSphere a mnoho ďalších.

\section{Štruktúra}
Liferay Portal umožňuje spravovať užívateľov a priraďovať im oprávnenia na jednotlivé časti portálu, alebo ich zaraďovať do rôznych typov skupín podľa oprávnení alebo záujmov. Nasledujúce podkapitoly stručne opisujú veľmi široké možnosti Liferay portálu.

\subsection{Používatelia}
\label{liferay_users}
Do portálov majú prístup užívatelia (Users), ktorý môžu byť priraďovaný do skupín (User Groups). Užívatelia taktiež môžu byť priraďovaný do organizácií (Organizations). Tieto organizácie sú zoskupované do rôznych hierarchických štruktúr, popri ktorých ešte v Liferay existujú komunity (Communities) a tímy (Teams). Detailnejší náhľad do štruktúry nám poskytne obrázok \ref{liferay_structure}.

Liferay tak v tomto smere poskytuje pre portálových administrátorov veľmi silný nástroj a taktiež obrovskú škálovateľnosť. Informácie o jednotlivých povoleniach (Permissions) vrámci systému sú pevne definované v rolách (Roles). Užívatelia, skupiny užívateľov, organizácie a komunity môžu byť priraďované do rolí\cite{sezov2010portal}.

\begin{figure}[H]
	\center
	\includegraphics[width=1.0\linewidth]{liferay_structure}
	\caption{Model štruktúry a oprávnení portálového servera Liferay\cite{sezov2010portal}}
	\label{liferay_structure}
\end{figure}

\subsection{Portlety}
Základnou stavebnou jednotkou portálu v Liferay je portlet. Každá stránka portálu sa skladá z portletov, pričom každý portlet je v kontexte Liferay samostatná aplikácia. Užívateľ je schopný umiestňovať tieto portlety ľubovoľne po stránke a nastavovať im parametre.

Liferay obsahuje veľké množstvo vopred pripravených portletov, avšak jeho hlavná výhoda tkvie v možnosti rozšíriť ponuku týchto portletov o svoje vlastné. Jednotlivé stránky obsahujúce portlety (pages) je potom možné zaraďovať do lokalít (sites).

\subsection{Stránky}
Stránky sú stavebné elementy v Liferay ktoré obsahujú portlety. Podľa viditeľnosti stránok pre používateľov delíme stránky na:
\begin{description}
\item[Verejné stránky] - sú viditeľné aj pre užívateľov, ktorí nie sú registrovaný v systéme
\item[Privátne stránky] - vyžadujú sa určité oprávnenia, ktoré musí mať užívateľ priradené
\begin{itemize}
\item privátna stránka používateľa - len autor má k nej prístup
\item stránka používateľskej skupiny - ak je užívateľ priradený do tejto skupiny, má prístup na stránku
\item privátna stránka portálu - každý prihlásený používateľ má prístup na stránku
\end{itemize}
\end{description}

\subsection{Lokality}
Podstatným elementom Liferay sú lokality (Sites), ktoré pomáhajú vytvárať organizovanú štruktúru stránok a používateľov. Na základe nastavení lokality potom môžeme spravovať práva prístupu pre jednotlivých užívateľov. Liferay podporuje tri základné typy lokalít\cite{burska2016portlety}:
\begin{description}
\item[Otvorené (Open)] - ľubovoľný používateľ sa môže stať členom danej lokality, stačí keď si danú lokalitu nájde v zozname
\item[Na požiadanie (Retricted)] - členstvo používateľa v danej lokalite musí byť schválené
\item[Súkromné (Private)] - lokalita nie je viditeľná bežnému užívateľovi ani v zozname, jeho členstvo musí byť manuálne nastavené administrátorom
\end{description}
Z podkapitoly \ref{liferay_users} však vieme, že lokality sú priradené organizácii a preto nastavenie oprávnení pre organizáciu platí rovnako pre všetkých užívateľov spadajúcich pod danú organizáciu. Aby lokalita mohla užívateľom zobrazovať nejaký obsah, musí obsahovať aspoň jednu stránku.

\chapter{Angular framework}
Angular je open-source platforma od spoločnosti Google určená pre tvorbu jednostránkových front-end aplikácií. Využíva TypeScript ako odporúčaný programovací jazyk, ale je možné využiť aj JavaScript.

\section{TypeScript}
TypeScript je programovací jazyk vyvýjaný firmou Microsoft. Prvý krát bol oficiálne predstavený v roku 2012 a je označovaný ako "syntaktický cukor" pridaný do JavaScriptu, ktorý umožňuje využívať typovú kontrolu.

\subsection{Porovnanie s JavaScriptom}
JavaScript je dynamicky typovaný jazyk. To znamená, že typ premennej záleží na samotnej hodnote danej premennej. V jednotlivých fázach aplikácie teda môžeme do premennej priradiť hodnoty rôznych typov. To nám v istom smere poskytuje veľkú slobodu, ale zároveň vnáša istú mieru nepresnosti a neurčitosti do kódu, čo môže hlavne v rozsiahlych aplikáciách spôsobiť problémy so spoľahlivosťou.

\begin{lstlisting}
private example(): void {
  let attribute = 'some_string';
  console.log(typeof attribute); // nam vrati "string"
  attribute = 54;
  console.log(typeof attribute); // nam vrati "number"
}
\end{lstlisting}

TypeScript tvorí nadstavbu nad JavaScriptom. Výsledkom je teda kód, ktorý sa vo výsledku aj tak preloží len do JavaScriptu, avšak vďaka tomuto prekladu získame typovú kontrolu. Okrem toho nám niektoré editory podporujúce TypeScript dokážu navrhovať a zvýrazňovať syntax na základe vopred nadefinovaných typov. Pre vyššie uvedený kód by TypeScriptový prekladač skončil s chybou.

Typescript teda môžeme považovať za staticky typovaný jazyk, avšak výzkum Kalifornskej univerzity\cite{ray2014large} ukázal, že približne 50\% všetkých použitých atribútov v bežnom TypeScript kóde má deklarovaný typ \textit{any}, pre ktorý prekladač typovú kontrolu nevykonáva. Je teda na mieste úvaha, či môžeme považovať typovú kontrolu v TypeScripte za plnohodnotnú. Môj názor je, že pri dodržiavaní správnych postupov a zvyklostí pri programovaní je táto kontrola zmysluplná.


\chapter{Návrh modulu}
Cieľom tejto práce je vytvoriť modul vo frameworku Angular, ktorý bude nasaditeľný ako samostatný portlet do Liferay portálu a prostredníctvom ktorého bude užívateľ schopný zvoliť si podmnožinu časového ohraničenia. Modul tak umožní užívateľovi spresniť (zúžiť) časové okno vrámci povolených hodnôt.

Vstupom tohto modulu bude
\begin{description}
\item[krok (step)] - najmenšia uvažovaná časová jednotka (získa sa volaním REST rozhrania)
\item[minimum, maximum] - dolné a horné časové ohraničenie (získa sa volaním REST rozhrania)
\item[IP adresa] na ktorú budú prebiehať REST volania z predchádzajúcich dvoch bodov (získa sa z konfigurácie portletu)
\item[príznak periodických volaní (\textit{LIVE})] - informácia o tom, či má modul periodicky overovať zmenu horného časového ohraničenia (získa sa z konfigurácie portletu)
\begin{description}
\item[perióda] - ak je príznak periodických volaní nastavený (\textit{true}), zoberie sa z konfigurácie portletu aj táto číselná hodnota (v~sekundách)
\end{description}
\end{description}
Výstupom tohto modulu budú dve časové známky, kde prvá odpovedá aktuálnej pozícii ľavého manipulátora a druhá pozícii pravého manipulátora.

\section{Požiadavky}
Požiadavky na dizajn a funkcionalitu som vytvoril na základe konzultácií s vedúcim práce RNDr. Radkom Ošlejškom, Ph.D.

Hlavnou úlohou časovej osy je
\begin{itemize}
\item zrozumiteľne zobraziť informáciu užívateľovi o aktuálne zvolenom časovom rozsahu
\item umožniť užívateľovi vybrať a upraviť časový rozsah pomocou vhodného nástroja (časovej osy s manipulátormi)
\end{itemize}

\subsection{Funkčné požiadavky}
Medzi funkčné požiadavky patria
\begin{itemize}
\item modul si zistí možné časové ohraničenie REST volaním verejného aplikačného rozhrania (API) pomocou HTTP požiadavku
\item užívateľ bude informovaný vhodnou hláškou ak sa REST volanie nepodarí, alebo príde nevyhovujúca odpoveď
\item časovú os bude možné spustiť v móde \textit{LIVE} dát pričom sa bude meniť len horné ohraničenie. Dynamickosť dát bude zabezpečovať samotný modul pomocou periodických volaní API vždy po uplynutí vopred definovanej doby.
\begin{itemize}
\item definovanie správania časovej osy v prípade zmeny ohraničenia bude implementované pomocou zámkov. Bude možné zamkýnať obe strany časovej osy. Požiadavky na funkcionalitu zámkov sú opísané v sekcii \ref{sec:lockers}~\nameref{sec:lockers}.
\end{itemize}
\item modul bude pozostávať z dvoch časových osí uložených nad sebou. Vrchná časová os bude vždy zobrazovať celé možné časové ohraničenie, zatiaľ čo spodná bude zobrazovať len práve zvolený výber.
\end{itemize}

\subsection{Nefunkčné požiadavky}
Medzi nefunkčné požiadavky patria
\begin{itemize}
\item modul bude pracovať s časovými známkami (timestamp), ktoré sú univerzálnym označením časového okamihu v UNIXových operačných systémoch. Výsledný modul tak nebude zohľadňovať časové pásma.
\item modul bude implementovaný v aktuálnej verzii frameworku Angular, v dobe implementácie sa jedná o verziu 5.2.8
\end{itemize}

\subsection{Funkcia zámkov}
\label{sec:lockers}
V prípade, že modul časovej osy bude pracovať s dátami v reálnom čase, horné ohraničenie sa logicky bude zväčšovať. V takomto prípade môžu nastať aj zmeny aktuálneho výberu, ktoré môžu ale nemusia byť pre užívateľa žiadúce.

Z tohto dôvodu vznikla už v minulosti požiadavka, aby sám používateľ portálu mal možnosť toto správanie definovať. Za týmto účelom bude pri oboch manipulátoroch vrchnej časovej osy klikateľný ukazateľ zámku, ktorého správanie opisuje nasledujúca tabuľka možností:
\begin{center}
  \begin{tabular}{ | c | c || p{6.5cm} | }
    \hline
    ľavý zámok & pravý zámok & správanie pri zmene horného ohraničenia \\ \hline \hline
    odomknutý & odomknutý & výber sa nemení, aktualizuje sa len možné horné ohraničenie viditeľné v hornej osy \\ \hline
    odomknutý & zamknutý & pravý manipulátor sa nastaví na maximum a ľavý sa posunie doprava tak, aby veľkosť zvoleného rozmedzia ostala zachovaná\\ \hline
    zamknutý & odomknutý & zakázaný stav - ľavý zámok môže byť aktivovaný len ak je aktivovaný zároveň aj pravý \\ \hline
    zamknutý & zamknutý & pravý manipulátor sa nastaví na maximum a ľavý zostáva, veľkosť zvoleného rozmedzia sa zväčší \\ \hline
  \end{tabular}
\end{center}


%\printbibliography

\printbibliography[heading=bibintoc] %% Print the bibliography.

  \makeatletter\thesis@blocks@clear\makeatother
  \phantomsection %% Print the index and insert it into the
  \addcontentsline{toc}{chapter}{\indexname} %% table of contents.
  \printindex

\appendix %% Start the appendices.
\chapter{An appendix}
Here you can insert the appendices of your thesis.

\end{document}